\documentclass{article}

\usepackage{ctex}

\newcommand{\myFont}{\textit{\Huge{myfont}}}

\title{\heiti 字体输入方法}% \heiti 黑体字体
\author{Wang QW}
\date{\today}


\begin{document}
	\maketitle
\section{中文输入}
	中文输入
\section{查看文档}	
	使用texdoc查看文档
	
	texdoc ctex
	
	texdoc lshort -zh
	
\section{字体设置}
\subsection{字体族}
%字体族设置
\textrm{Roman Family} \textsf{Sans Serif Family}
\subsection{作用范围}
%作用范围
{\rmfamily Roman in a big brace}

\rmfamily This is a paragraph,and you can see in this paragraph the word is in a form of Roman Family.

%字体形状设置
\subsection{字体形状}
\textup{直立} \textit{Italic Shape} \textsl{Slandted Shape} \textsc{Small Caps Shape}
\subsection{中文字体}
{\songti 宋体} \quad{\heiti 黑体} \quad{\kaishu 楷书}
中文字体的\textbf{粗体}与\textit{斜体}
\subsection{字体大小}
%字体大小设置
{\tiny Hello}\\
{\scriptsize Hello}\\
{\footnotesize Hello}\\
{\small Hello}\\
{\normalsize Hello}\\
{\large Hello}\\
{\Large Hello}\\
{\LARGE Hello}\\
{\huge Hello}\\
{\Huge Hello}

%中文字号
\zihao{5} 你好

\myFont


\end{document}